\documentclass[11pt,a4paper,oneside,draft]{article}

\setlength{\columnsep}{20pt}
\usepackage{ifdraft}

\usepackage[margin=1cm, bottom=2cm]{geometry}
\usepackage{amsthm, amsmath, amssymb, latexsym,  enumerate, cleveref, xfrac}
\def\rmdefault{ptm}

\usepackage{alltt}

\theoremstyle{definition}
\newtheorem{definition}{Definition}[section]
\newtheorem{proposition}[definition]{Proposition}
\newtheorem{theorem}[definition]{Theorem}
\newtheorem{example}[definition]{Example}
\newtheorem{corollary}[definition]{Corollary}
\newtheorem{lemma}[definition]{Lemma}
\theoremstyle{remark}
\newtheorem{remark}[definition]{Remark}
\newtheorem{notation}[definition]{Notation}

\newcommand{\bi}{\begin{itemize}
\setlength{\itemsep}{1pt}
  \setlength{\parskip}{0pt}
  \setlength{\parsep}{0pt}}
\newcommand{\ei}{\end{itemize}}
\newcommand{\be}{\begin{enumerate}
\setlength{\itemsep}{1pt}
  \setlength{\parskip}{0pt}
  \setlength{\parsep}{0pt}}
\newcommand{\ee}{\end{enumerate}}
\newcommand{\bd}{\begin{description}}
\newcommand{\ed}{\end{description}}

\newcommand{\lp}{{\cal L}}

\newcommand{\g}{\Gamma}
\def\d{\Delta}
\newcommand{\su}{\supset}
\newcommand{\w}{\wedge}

\newcommand{\st}{{\ : \ }}   % such that

\newcommand{\fv}[1]{{Fv[#1]}}

\newcommand{\nin}{\not\in}
\newcommand{\nmodels}{\not\models}

\newcommand{\vd}{\vdash}

\newcommand{\bs}{\bigskip}
\newcommand{\ms}{\medskip}
\newcommand{\til}{,\dots,}
\newcommand{\tup}[1]{{\langle #1 \rangle}}
\newcommand{\dom}[1]{{\vert #1 \vert}}

\newcommand{\fe}{\varphi}
\newcommand{\ra}{\rightarrow}

\newcommand{\suq}{\subseteq}

\usepackage[usenames]{color}
\newcommand{\ori}[1]{{{{\textcolor{blue}{Ori: {\em #1}}}}}}


\makeatletter
\def\stack#1#2{%
  \mathchoice
    {\@stack{\displaystyle}{#1}{#2}}%
    {\@stack{\textstyle}{#1}{#2}}%
    {\@stack{\scriptstyle}{#1}{#2}}%
    {\@stack{\scriptscriptstyle}{#1}{#2}}%
}
\newdimen\@stackwda
\newdimen\@stackwdb
\def\@stack#1#2#3{%
  \settowidth{\@stackwda}{$#1 #2$}%
  \settowidth{\@stackwdb}{$#1 #3$}%
  \advance\@stackwda-\@stackwdb
  \@stackwda=.5\@stackwda
  \everymath{#1\everymath{}}%
  \renewcommand{\arraystretch}{.6}%
  \setlength{\arraycolsep}{0em}%
  \ifnum\@stackwda<\z@ \hskip\@stackwda\fi%
  \begin{array}[t]{c}
    #2\\#3
  \end{array}
  \ifnum\@stackwda<\z@ \hskip\@stackwda\fi%
}
\makeatother

\title{Extensions of CAV}

\begin{document}

\maketitle


\section{Cycles extension - copied from the CAV paper}

\newcommand{\nextf}{\textit{next}}
\newcommand{\B}[1]{\langle{#1}\rangle}
\newcommand{\Lor}{\bigvee}
\newcommand{\AF}{\mbox{$\textit{AF}^R$}}
\newcommand{\tabref}[1]{Table~\ref{Ta:#1}}
\newcommand{\figref}[1]{Fig.~\ref{Fi:#1}}

Instead of keeping track of just $\nextf^*$, we instrument the edge
addition operation with a check: if the added edge is about to close
a cycle, then instead of adding the edge, we keep it in a separate relation $m$ of
``cycle-inducing'' edges. Two properties of lists now come into play:
%\begin{enumerate}
%  \item 
(1)~The number of cycles reachable from program variables, and hence the size of $M$, is bounded by the number of program variables;
(2)~Any path (simple or otherwise) in the heap may utilize at most one
    of those edges, because once a path enters a cycle, there is no way out.
%\end{enumerate}
In all assertions, therefore, we replace $\alpha\B{\nextf^*}\beta$ with:
%
%\begin{center}
$\alpha\B{\nextf^*}\beta \lor
   \Lor_{\langle u,v\rangle\in M}
      (\alpha\B{\nextf^*}u \land v\B{\nextf^*}\beta)$.
%\end{center}
Notice that it is possible to construct this formula thanks to the bound on
the size of $M$; otherwise, an existential quantifier would have been required
in place of the disjunction.

\medskip
Cycles can also be combined with nesting, in such a way as to introduce
an unbounded number of cycles. To illustrate this, consider the example
of a linked list beginning at $h$ and formed by a pointer field which we
shall denote $n$, where each element serves as the head of a singly-linked
cycle along a second pointer field $m$. This is basically the same as
in the case of acyclic nested lists, only that the last node in every
sub-chain (a list segment formed by $m$) is connected to the first node
of that same chain.

One way to model this in a simple way is to assume that the programmer
designates the last edge of each cycle; that is, the edge that goes back
from the last list node to the first.
We denote this designation by introducing a ghost field named $c$. This
cycle-inducing edge is thus labeled $c$ instead of $m$.

Properties of the nested data structure can be expressed with $\AF$ formulas
as shown in \tabref{nested-cyclic-list-props}. ``Hierarchy'' means
that the primary list is contiguous, that is, there cannot be $n$-pointers
originating from the middle of sub-lists. ``Cycle edge'' describes the closing
of the cyclic list by a special edge $c$.

We were able to verify the absence of memory errors and the correct 
functioning of the program {\tt flatten}, shown in \figref{flatten}.

\bgroup

\newcommand\via[1]{\langle #1 \rangle}

\newcommand\vianpre{\via{\underline{n}}}
\newcommand\vianrtcpre{\via{\underline{n}^*}}
\newcommand\vianrtc{\via{n^*}}
\newcommand\viamrtc{\via{m^*}}
\newcommand\viarrtc{\via{r^*}}

\renewcommand\a{\alpha}
\renewcommand\b{\beta}
\renewcommand\c{\gamma}

\newcommand\nn[2]{#1\vianrtc #2}
\newcommand\mm[2]{#1\viamrtc #2}
\newcommand\clink[2]{#1\via{c}#2}

\begin{table}
\centering
\begin{tabular}{|l|l|}
\hline
  All lists are acyclic & $sll(n^*) \land sll(m^*)$ \\
\hline
  No sharing between lists &
   $\begin{array}{@{}l@{\;}l}
      \forall\a,\b,\c\!:  &\nn h\a \land \mm\a\b \;\land\\ 
                          &\nn h\c \land \mm\c\b \implies \a=\c
    \end{array}$ \\
\hline
  Hierarchy &
   $\forall \a,\b,\c:  \a\neq\b \land \b\neq\c \land \mm\a\b \implies \lnot\nn\b\c$ \\
\hline
  Cycle edge &
   $\begin{array}{@{}l@{\;}l}
      \forall \a,\b,\c: \a\neq\b \land \a\neq\c \land \nn\a\b \implies \lnot \clink\a\c \\
      \forall \a,\b: \clink\b\a \implies \nn h\a \land \mm\a\b
    \end{array}$ \\
\hline
\end{tabular}
\caption{\label{Ta:nested-cyclic-list-props}Properties of a list of cyclic lists expressed in $\AF$}
\end{table}

\begin{figure}
\begin{alltt}\small
\begin{tabbing}
No\=de flatten(Node h) \{\+\\
Node i = h, j = null;\\
wh\=ile (i != null) I1 \{\+\\
Node k = i;\\
wh\=ile (k != null) I2 \{\+\\
j = k; k = k.m;\-\\
\}\\
j.c = null;\\
i = i.n; j.m = null; j.m = i;\-\\
\}\\
j.c = null; j.c := h;\\
return h;\-\\
\}
\end{tabbing}
\end{alltt}
\caption{\label{Fi:flatten}A program that flattens a hierarchical structure of lists into a single cyclic list.}
\end{figure}

\newpage

\section{Cycles}
\begin{itemize}
\item The commands speak about $n$ field, and $\tup{n^*}$ is used in the assertions.

\item Behind the scenes we have an acyclic relation $\langle k^* \rangle$,
and an auxiliary binary relation denoted by $\tup{m}$.

\item $\alpha \tup{n^*} \beta$ in the assertions is translated to: $\alpha \tup{k^*} \beta \lor \vec{nk}(\alpha) \tup{k^*} \beta$.

\item $\tup{k^*}$ is axiomatized with $\Gamma_{lin}$ as in the CAV paper.
\item Extra Axioms:
\begin{itemize}
\item $(A)$ $\forall \alpha,\beta,\gamma.~ \alpha \tup{k^*} \beta \land \alpha  \tup{m} \gamma \to
\alpha=\beta$
(there cannot be $k$ and $m$ from the same node)
\item $(B)$  $\forall \alpha,\beta,\gamma.~ \alpha \tup{m} \beta \land \alpha \tup{m} \gamma \to
\beta=\gamma$
\item $(C)$  $\forall \alpha,\beta.~ \alpha \tup{m} \beta \to \beta \tup{k^*} \alpha$
\end{itemize}

\item Additionally, two unary function symbols are used: $\vec{k}$ and $\vec{km}$.
\begin{itemize}

\item $\vec{k}( \alpha)$: 
\begin{itemize}
\item Intended meaning: the last node reachable from $ \alpha$ via $n$.
\item Axiom: $\forall \alpha,\beta.~ \alpha \tup{k^*} \beta \to \beta  \tup{k^*} \vec{k}( \alpha)$
\end{itemize}

\item $\vec{km}( \alpha)$: 
\begin{itemize}
\item Intended meaning: the node reachable from $\vec{k}( \alpha)$ via $m$.
\item Axiom: 
$\forall \alpha,\beta,\gamma.~ \alpha \tup{k^*} \beta \land \beta \tup{m} \gamma  \to \vec{km}(\alpha) = \gamma$
\item Axiom: $\forall \alpha.~  \vec{k}(\alpha) \tup{m} \vec{km}(\alpha) ~ \lor ~ \vec{km}(\alpha) = \vec{k}( \alpha)$

\end{itemize}

Their properties ensure they can be used in EPR (these properties {\bf follow} from the previous ones and the general theory for $\tup{k^*}$):
\begin{itemize}
\item Idempotence: $\forall \alpha.~\vec{k}(\vec{k}(\alpha))=\vec{k}(\alpha)$ 
\item Idempotence: $\forall \alpha.~\vec{km}(\vec{km}(\alpha))=\vec{km}(\alpha)$
\item $\forall \alpha.~\vec{km}(\vec{k}(\alpha))=\vec{km}(\alpha)$ 
\item $\forall \alpha.~\vec{k}(\vec{km}(\alpha))=\vec{k}(\alpha)$ 
\end{itemize}
\end{itemize}


\item A predicate $Between(\alpha,\beta,\gamma)$ standing for ``there is a simple path
from $\alpha$ to $\gamma$ through $\beta$" can be defined by:
$$Between(\alpha,\beta,\gamma):= (\alpha \tup{k^*} \beta \tup{k^*} \gamma)
\lor (\vec{km}(\alpha) \tup{k^*} \beta \tup{k^*} \gamma)
\lor (\alpha \tup{k^*} \beta \tup{k^*} \vec{k}(\alpha) \land \vec{km}(\alpha) \tup{k^*} \gamma)$$

Using $Between(\alpha,\beta,\gamma)$ only, the programmer is not aware of our internal troubles with $k,m$ and $\vec{km},\vec{k}$.

\item A predicate $OnCycle(\alpha)$ standing for ``$\alpha$ is on a cycle" is defined by:
$(\vec{k}(\alpha) \tup{m} \vec{km}(\alpha) \land \vec{km}(\alpha) \tup{k^*} \alpha)$

\item To compute the weakest pre-conditions of $x.n := null$ use a microcode:

1: if  $OnCycle(x)$ \\
2: \qquad $s := \vec{k}(x)$ \\
3: \qquad $t := \vec{km}(x)$ \\
4: \qquad $x.k := null$ \\
5: \qquad $s.m := null$ \\
6: \qquad if $s \neq x$ \\
7: \qquad \qquad  $s.k := t$ (we may assume that $s.k = null$)\\
8: else \\
9: \qquad $x.k := null$


\item To compute the weakest pre-conditions of $x.n := y$ use a microcode (assuming that $x.n := null$ was performed immediately before)

1: if  $y \tup{k^*} x$ \\
2: \qquad $x.m := y$   (we may assume that $x.m = null$)\\
3: else \\
4: \qquad $x.k := y$ (we may assume that $x.k = null$)\\


\item To compute the weakest pre-conditions of $x := y.n$ use a microcode

%1: if  $(\vec{k}(y) = y \land y \neq  \vec{km}(y)) \lor y  \tup{m} y$ \\
1: if  $\vec{k}(y) = y$ \\
2: \qquad $x := y.m$ \\
3: else \\
4: \qquad $x := y.k$


\item additions to $wp$ for function symbols:

\begin{itemize}
\item $x.k := y$ (assuming that $x.k = x.m = null$):
\begin{itemize}
\item substitute $\vec{k}(\alpha)$ by $ite(\vec{k}(\alpha)=x, \vec{k}(y), \vec{k}(\alpha))$
\item substitute $\vec{km}(\alpha)$ by $ite(\vec{k}(\alpha)=x, \vec{km}(y), \vec{km}(\alpha))$
\end{itemize}
\item $x.m := y$ (assuming that $x.k = x.m = null$):
\begin{itemize}
\item no change in $\vec{k}(\alpha)$
\item substitute $\vec{km}(\alpha)$ by $ite(\vec{k}(\alpha)=x, y, \vec{km}(\alpha))$
\end{itemize}
\item $x.k := null$:
\begin{itemize}
\item substitute $\vec{k}(\alpha)$ by $ite(\alpha \tup{k^*} x, x, \vec{k}(\alpha))$
\item substitute $\vec{km}(\alpha)$ by $ite(\vec{k}(x)\neq x \land \alpha \tup{k^*} x, x,\vec{km}(\alpha))$
\end{itemize}
\item $x.m := null$
\begin{itemize}
\item no change in $\vec{k}(\alpha)$
\item substitute $\vec{km}(\alpha)$ by $ite(\vec{k}(\alpha)=x , x, \vec{km}(\alpha))$
\end{itemize}

\end{itemize}




\end{itemize}

\section{Nested Linked List}

\begin{itemize}

\item The commands speak about $d$ (down) and $r$ (right) field.
\item The assertions may include $\tup{d^*}$, $\tup{r^*}$, $\tup{(d\cup r)^*}$.

\item This only works when there is no sharing ---
if $r(\alpha)=\beta$ then: $r(\gamma)=\beta$ iff $\gamma=\alpha$, and $d(\gamma)\neq\beta$ for every $\gamma$.
\begin{itemize}
\item Add to the pre-condition of $x.r := y$ the fact that $y$ does not become shared:
$\forall \alpha. ~ (\alpha \tup{d^*} y \to \alpha=y) \land (\alpha \tup{r^*} y \to \alpha=y)$. 
Make this a conjunct in the $wp$ of $x.r := y$.

\item Add to the pre-condition of $x.d := y$ the fact that $y$ does not become shared:
$\forall \alpha. ~ (\alpha \tup{r^*} y \to \alpha=y)$. 
Make this a conjunct in the $wp$ of $x.d := y$.

\end{itemize}

\item One unary function symbol are used: $\overleftarrow{r}$.
\begin{itemize}
\item $\overleftarrow{r}(\alpha)$: 
\begin{itemize}
\item Intended meaning: the first node that can reach $ \alpha$ via $n$.
\item Axiom: $\forall \alpha,\beta. \beta \tup{r^*} \alpha \to \overleftarrow{r}(\alpha) \tup{r^*} \beta$
\item Idempotence: $\forall \alpha.~\overleftarrow{r}(\overleftarrow{r}(\alpha))=\overleftarrow{r}(\alpha)$ 
\end{itemize}


\end{itemize}

\item $\alpha \tup{(d\cup r)^*} \beta$ in the assertions is translated to: $\alpha \tup{n^*} \overleftarrow{r}(\beta)$.



\end{itemize}





\end{document}